\begin{question}[section=3,name={Anfangszustand},difficulty=5,type=mdl,tags={}]
	Gegeben ist ein System in folgender Darstellung: $\dot{\vec x} = \underline A\vec x + \underline B\vec u$, $y = \underline C\vec x + \underline D\vec u$ mit dem Anfangszustand $\vec x_0$. Geben Sie die allgemeine Lösung an. (im Zeitbereich und im Laplacebereich + Herleitungen)
	\\ \textbf{Hinweis:}\\
	Satz 2.4
\end{question}
\begin{solution}
	\begin{align}
		s \vec X(s) -\vec x_0 & = \underline A \vec X(s) + \underline B \vec U(s)\\
		\vec X(s) (s \underline E - \underline A) &= \vec x_0 + \underline B \vec U(s)\\
		\vec X(s) &= (s \underline E - \underline A)^{-1}(\vec x_0 + \underline B \vec U(s))\\
		\vec Y(s) &= \underline C \vec X(s) + \underline D \vec U(s)\\
		\vec Y(s) &= \underline C (s \underline E - \underline A)^{-1}\vec x_0 + (\underline C(s \underline E - \underline A)^{-1}\underline B +\underline D) \vec U(s)\\
		\mathscr L ^{-1}\{ \vec Y(s) \}&= \underline C \Phi(t)\vec x_0 + \underline C \int \limits _0 ^t\Phi(t-t')\underline B \vec u(t') dt' +\underline D \vec u(t)
	\end{align}
\end{solution}
\begin{question}[section=2,name={Transitionsmatrix},difficulty=3,type=mdl,tags={}]
	Transitionsmatrix und Eigenschaften
	\\ \textbf{Hinweis:}\\
	Buch Seite 28
\end{question}
\begin{solution}
	Die Transitionsmatrix kann über die Laplace-transformierte von
	\begin{equation}
		\Phi(t)=\mathscr L^{-1}\{ (s\cdot \underline E - \underline A)^{-1}\}
	\end{equation}
	bestimmt werden. Oder direkt mittels $e^{\underline A \cdot t}$ wobei hier $\underline A$ in diagonalform vorliegen muss.
	Dazu werden die Eigenwerte und Eigenvektoren von $\underline A$ bestimmt und die Dynamikmatrix in ihre Diagonalfrom Transformiert. $\tilde{\underline{A}}=\underline V ^{-1} \underline{A} \underline{V}$ Aus der Diagonalform wird dann die Transitionsmatrix gebildet, welche dann zurücktransformiert wird. $\Phi (t) = \underline{V} e^{\tilde{\underline{A}}t} \underline V ^{-1}$
	Kommen in der Diagonalform Jordanblöcke vor, müssen die wie folgt umgeformt werden:
	\begin{align}
		\left ( \begin{array}{ccc} \lambda_1 & 1 & 0 \\0 & \lambda_1 & 0\\ 0 & 0 & \lambda_2 \end{array} \right ) &\Leftrightarrow \left ( \begin{array}{ccc} e^{\lambda_1 t} & t\cdot e^{\lambda_1 t} & 0 \\0 & e^{\lambda_1 t} & 0\\ 0 & 0 & e^{\lambda_2 t} \end{array} \right )\\
		\left ( \begin{array}{cc} \alpha_1 & \beta_1  \\-\beta_1 & \alpha_1 \end{array} \right ) &\Leftrightarrow \left ( \begin{array}{cc} e^{\alpha_1 t} \cos(\beta_1 t) &  e^{\alpha_1 t}\sin(\beta_1 t)  \\ -e^{\alpha_1 t}\sin(\beta_1 t)& e^{\alpha_1 t} \cos(\beta_1 t)   \end{array} \right )
	\end{align}
	Eigenschaften der Transitionsmatrix:
	\begin{align}
		\underline{\Phi}(0) &= \underline E\\
		\underline{\Phi}(t+s) &= \underline{\Phi}(t)\underline{\Phi}(s)\\
		\underline{\Phi}(t)^{-1} &=\underline{\Phi}(-t)\\
		\frac{\partial \underline{\Phi}(t)}{\partial t} &= \underline A \cdot \underline{\Phi}(t)
	\end{align}
\end{solution}
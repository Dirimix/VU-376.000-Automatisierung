\begin{question}[section=2,name={Anfangszustand},difficulty=3,type=mdl,tags={}]
	Gegeben ist $\dot{\vec x} = \underline A \vec x$, ein Eigenwert $\lambda = 1$ und ein Eigenvektor in $x_1-x_2$-Ebene mit einem Anfangszustand $\vec x_0$ auf dem Vektor. Was kann man über das System aussagen?
	\\ \textbf{Hinweis:}\\
	Buch Seite 54
\end{question}
\begin{solution}
	Der Anfangszustand lässt sich dann aus einer Linearkombination der Eigenvektoren angeben.
	$\vec x_0 = \gamma_1 \vec v_1+\gamma_2 \vec v_2$
	Es entsteht eine Eigenschwingung und $\vec x (t)$ lässt sich  als $\vec x (t)=\gamma_1 e^{\lambda_1 t} \vec v_1 + \gamma_2 e^{\lambda_2 t} \vec v_2$ angeben. Der Eigenvektor muss invariant gegenüber $\underline A$ sein.
\end{solution}
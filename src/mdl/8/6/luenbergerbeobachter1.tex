\begin{question}[section=8,name={Luenberger Beobachter 1},difficulty=,type=mdl,tags={}]
	Wie sieht ein vollständiger Luenberger Beobachter aus, was ist die Fehlerdynamik (Herleitung), wo möchte ich die Eigenwerte der Fehlerdynamik liegen haben (Im Inneren des Einheitskreises) und wann kann ich die Eigenwerte frei platzieren und wie?
	\\ \textbf{Hinweis:}\\
	
\end{question}
\begin{solution}
	Wenn System vollständig beobachtbar ist, können die Eigenwerte frei plaziert werden. Durch Ackermann, bzw. Polvorgabe direkt wenn es in der Beobachtbarkeitsnormalform vorliegt.
\end{solution}